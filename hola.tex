\documentclass[10pt]{article}
% depende del tipo de archivo \documentclass[10pt]{book}
\usepackage[T1]{fontenc}
\usepackage[utf8]{inputenc}
\usepackage[spanish]{babel}

\begin{document}
% chaper* {} es el no numerado
% \chapter{Aspectos Generales} %
\title{Practica 2.\\ Texto normal, párrafos y alineación}
\author{Betzabe Machaca}
% sin fecha 
\date{}
\maketitle
\tableofcontents
\newpage
\section{Tipos y tamaños de fuente}
Segunda practica de redactar un articulo, sección para revisar tipos y tamaños de fuente a usar.
\subsection{Tipos de letras}
\begin{itemize}
\item Tipo de letra \textbf{negrita}
\item Tipo de letra \textit{itálica}
\item Tipo de letra \textrm{romana}
\item Tipo de letra \textsf{sans serif}
\item Tipo de letra \texttt{mono espaciada}
\item Tipo de letra \textsl{inclinada}
\item Tipo de letra \textsc{versalitas}
\end{itemize}
% salto de página \newpage
\subsection{Tamaños de letra}
\begin{itemize}
\item {\tiny Tamaño} de letra
\item {\scriptsize Tamaño} de letra
\item {\footnotesize Tamaño} de letra
\item {\small Tamaño} de letra
\item {\normalsize Tamaño} de letra
\item {\large Tamaño} de letra
\item {\Large Tamaño} de letra
\item {\LARGE Tamaño} de letra
\item {\huge Tamaño} de letra
\item {\Huge Tamaño} de letra
\end{itemize}
\section{Párrafos, sangría y saltos de linea.}
\subsection{Sangría}
Segunda practica de redactar un articulo, sección para revisar tipos y tamaños de fuente a usar.Segunda practica de redactar un articulo, sección para revisar tipos y tamaños de fuente a usar. Y modificaremos la sangría del documento, aquí iniciaremos con sangría. 

\noindent Esta linea no queremos que tenga sangría.

Esta linea queremos que tenga sangría.
\subsection{Salto de línea y nueva página}
Segunda practica de redactar un articulo, sección para revisar tipos y tamaños de fuente a usar.\\Segunda practica de redactar un articulo, sección para revisar tipos y tamaños de fuente a usar.
\par 
Segunda practica de redactar un articulo, sección para revisar tipos y tamaños de fuente a usar. \\[0.5cm]
Pero esta orea que tenga espaciado.
\subsection{Alineación de párrafos}
\begin{flushleft}
Texto a la izquierda
\end{flushleft}
\begin{center}
Texto centrado
\end{center}
\begin{flushright}
Texto a la derecha
\end{flushright}
\subsection{Comillas y puntos suspensivos}
Este es un ejemplo : ``para comillas'', `simples' y las comillas ``dobles''.\\
Y también los puntos suspensivos \dots
\subsection{Espaciado Horizontal}
\noindent Inicio \, fin\\[0.2 cm]
Inicio \quad fin\\[0.2 cm]
Inicio \qquad fin\\[0.2 cm]
Inicio \hspace{3cm} fin\\[0.2 cm]
Inicio \hspace{5cm} fin\\[0.2 cm]
Inicio \hfill fin\\[0.2 cm]
\subsection{Lineas de relleno}
\noindent Inicio \hfill fin\\[0.5cm]
Inicio \hrulefill fin\\[0.5cm]
Inicio \dotfill fin\\[0.5cm]







\end{document}